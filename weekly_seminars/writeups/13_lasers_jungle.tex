\section{Lasers in the jungle: new measurments of structure and function, from tropical forests to city churchyards.}

\textbf{Speaker:} Dr Mathias Disney (\textit{University College London})\\
\textbf{Date:} \hspace{.53cm} 3\textsuperscript{rd} May 2018
\vspace{.5cm}

Lidar is the future! Using a device mounted on a terrestrial tripod they are able to very accurately capture and reproduce graphically the 3D structure of a forest. The example shown was that of Wytham wood outside Oxford. From this complex 3D structure they are able to get much more accurate measures of the above ground biomass. Historically this has come from either satellites or crude allometric equations based or even destructive methods in which they cut down the tree and actually weigh it. These methods both very inaccurate and incredibly expensive or time consuming.

Terrestrial laser scanning allows for accurate measures of volume and so above ground biomass as well as providing insights into tree branching length and limits on tree height. Initial scans return a dense cluster of points and so some clever computer algorithms are needed to turn these points into discrete volumes -- trees. While testing out this technology they are focussing mainly on sites which are already very well understood for their ecology -- making understanding their results much easier. Evidence suggests that their measures of biomass match pretty well when compared with the destructive method of cutting down and weighing trees. This method is also advantageous as it removes any chance of bias, all trees are sampled.

Another interesting outcome from this work is a realisation of how much tree branching occurs. For example the team LIDARed an Oak in Judi Dench's garden as part of project with the BBC. They found 12.5km of branching on what is a fairly usual oak tree -- similar results are found across all trees! They also produced a 3D model of the tree and allowed people to explore it in virtual reality which seemed pretty cool!