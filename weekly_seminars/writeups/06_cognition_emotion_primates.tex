\section{Studying cognition-emotion interaction in non-human primates using cognitive bias tasks; emerging trends and future directions.}

\textbf{Speaker:} Dr Emily J Bethell (\textit{Liverpool John Moores University})\\
\textbf{Date:} \hspace{.53cm} 23\textsuperscript{rd} November 2017
\vspace{.5cm}

The cognitive theory of emotion: Cognition is the processing of information in the the brain (including perception), Emotion is the response to stimulus. There are 2 routes to processing info in the brain, high road cognitive response and low road emotional response. 

The cognitive theory of emotion in humans. There is a feedback loop in emotional disorders - if you are in a poor emotional state then stimuli are processed negatively. The interaction between emotion and cognition is well established. Anxious people make negative judgments with ambiguous information, focussing more on negative and threatening stimuli - in an experiment they were quicker to notice threats (snake in the grass etc). Human studies provide many lessons we can adapt to other species.

Do monkey exhibit cognition-emotion interactions as humans do? Attention bias was found to be opposite to humans with anxious monkeys displaying avoidant behaviour - after being stressed the monkeys would look away/ignore threatening faces. However the results of this experiment did have a very high variability. Current work is staring to look at genetic factors and the impacts of early maternal separation.
In summary there is an increasing amount of evidence pointing to primates displaying cognition-emotion interactions. This is influenced by both genetic and environmental factors.