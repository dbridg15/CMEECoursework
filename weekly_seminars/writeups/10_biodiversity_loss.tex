\section{Why do the consequences of biodiversity loss diverge from expectation?}

\textbf{Speaker:} Prof. Martin Solan (\textit{University of Southampton})\\
\textbf{Date:} \hspace{.53cm} 8\textsuperscript{th} February 2018
\vspace{.5cm}

Half a billion years ago the majority of life on Earth was algal grazers. The evolution of life which penetrated the sediment layer led to a significant diversification of life as the nutrients within these sediments was released back into the water column and fuelled primary productivity. Today more than 30,000km\textsuperscript{2} of sediment in turned over everyday (that's a lot of sediment!). Soils and sediments are an extension of each other and form the largest habitat on Earth.

When the biodiversity of an ecosystem is altered there can be a variety of consequences. The NULL hypothesis would be that each species contributes equally to ecosystem function and so an linear relationship. However there are many other possibilities: there could be a single or few key species which or it could be stochastic. The order in which species are lost could also have a significant impact. 

\textbf{Random \textit{vs} Ordered Extinctions: } They did a few simulations to test this and found that loosing species by body-size is worse than random (i.e. larger sized species contribute more). They also found that loosing species by rarity (with rarest first) was not as bad as random extinctions -- suggesting rare species contribute little to ecosystem functioning. Clearly the order of extinction events does have an impact! 	They tested this experimentally with mudflats. Mudflats are ideal for this experiment and the only have 3-4 species it is possible to test each possible order of extinctions. 