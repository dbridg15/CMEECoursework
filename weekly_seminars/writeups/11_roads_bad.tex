\section{Can nature survive the global infrastructure tsunami?}

\textbf{Speaker:} Prof. William F. Laurance (\textit{James Cook University})\\
\textbf{Date:} \hspace{.53cm} 2\textsuperscript{nd} March 2018
\vspace{.5cm}

The last few decades have seen an unheard-of expansion in infrastructure (roads, power-lines, dams, fossil fuel projects) especially in developing nations such as China and South America. Much of this expansion is poorly planned out with little thought on its environmental impact. The impact on the environment is increased by the fact that these developing nations are often the most biodiverse.

Roads have a multitude of effects. Direct effects: such as habitat loss, edge effects where the climate/environment is harsher at the habitat edges which are created. Road kill is also a problem and barrier effects where roads divide individuals territories is having an impact on bird behaviour. Roads also act as a corridor for invasion with mosquitos, cane toads and fire ant all utilising roads to spread. Indirect effects are also a problem: the presence of roads increases the amount of illegal deforestation/poaching/gold mining that can occur. \textbf{In summary roads are bad!}

Beyond the impact of the initial road there are also problems with infrastructure projects. These start with the corruption and negative impact on local economies and end with poorly built roads which need significant maintenance if they are going to be useful.