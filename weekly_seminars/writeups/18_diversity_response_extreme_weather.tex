\section{Predicting changes in the diversity of functionally important plants in response to extreme weather: How far can we go with statistical species Niche Models?}

\textbf{Speaker:} Dr Simon Smart (\textit{NERC Centre for Ecology \& Hydrology})\\
\textbf{Date:} \hspace{.53cm} 21\textsuperscript{st} June 2018
\vspace{.5cm}

The aim of this study was to determine how key service-providing plants respond to extreme weather events and gradual climate change. There are multiple of global biodiversity change - detailed in Sala et al (2000) `Global Biodiversity Scenarios for the Year 2100' all affecting the biodiversity and composition of ecosystems.

This study looked  at the impacts and response to flooding (after storm Desmond) in high quality agricultural land and hill land/river valley (Lyth Valley) valued by wildlife and for its natural beauty. They are trying to answer questions such as: Will extreme weather with climate change lead to colonization failures due to the estimated vacant nice space? and Will potential colonists deliver a different set of services than existing vegetation.

Statistical species niche model: key axis which define where a plant lives, for example; shade, disturbance, fertility, pH. Using the R package MultiMOVE they made predictions on the species composition of ecosystems under different climatic and weather stresses. The package covers all nectar plants, all dominant ecosystem species as well as a breadth of rarer species, with each niche model being the weighted average of 5 models (MARS, GLM, Random Forests, GAM and Neural Networks). They predicted for Lyth Valley, negative impacts for all but one group and found that a warmer, drier climate buffers the impact of flooding but increases the risk of drought.

There are a number of criticisms of the model which he addresses. Space $\neq$ time. No dynamic and so no novel outcomes. Models depict niches of the past with possible unknown interactions.