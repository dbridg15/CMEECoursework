\section{Local adaption in humans and other primates.}

\textbf{Speaker:} Dr Aida Andres (\textit{University College London})\\
\textbf{Date:} \hspace{.53cm} 14\textsuperscript{th} June 2018
\vspace{.5cm}

Local adaption is the adaption of a population to be better suited to local conditions - while remaining part of the same species, local adaption occurs on relatively short timescales. Humans are an excellent example of local adaption, having colonised to the majority of the globe: moving out of Africa about 50,000 years ago and now living comfortably in all kinds of environmental conditions. By comparing populations from Europe and  Africa it is possible to determine which alleles differ and so may have been impacted by selection pressure - leading to local adaption. Using an ancient `out-of-Africa' genome it is possible to eliminate alleles which were fixed due to drift - as drift occurs at a much faster rate than selection and so this early genome will have fixed alleles due to drift but will not have had time for local adaption to occur when compared to a modern day `out-of-Africa' genome. There is evidence that local adaption significantly contributed to strong allele frequency population differentiation.

Adaption to ambient temperature is an obvious trait to study as average annual temperature varies so much globally; around \SI{28}{\celsius} in Africa and \SI{6}{\celsius} in Finland. Cold response has also been shown to be impacted by a single gene controlling a Transient Receptor Potential (TRP) ion channel. They found that the cold adapted allele had a much higher frequency in higher latitudes with colder temperature and actually found a significant correlation between the allele frequency and latitude/temperature.


