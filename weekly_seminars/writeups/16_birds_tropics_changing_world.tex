\section{Monitoring birds and tropical forest in a changing world.}

\textbf{Speaker:} Dr Martin Sullivan (\textit{University of Leeds})\\
\textbf{Date:} \hspace{.53cm} 7\textsuperscript{th} June 2018
\vspace{.5cm}

The world and its environment is changing rapidly on different spatial and temporal scales. 
Long-term monitored datasets are a vital resource to better understand biological responses to these changes and predict how these changes will progress in the future. 

\textbf{UK Birds:}
Examples of long term datasets include the monitoring of grey herons in the UK (since 1984) and the `Uk breeding bird survey' which has run since 1994 with around 2800 volunteers monitoring bird populations at over 3500 1km grid squares.
Different species are faring differently with some species' populations decreasing while others are increasing. This may be related to functional traits or habitat, with the fate of a species being strongly linked to the type of habitat they prefer, explained by density dependence. Generalist species are doing better in the UK now compared with specialist species.

\textbf{Change of tack to the rainforest:}
Rainforests are home to >50\% of terrestrial biodiversity and a crazy amount of carbon. Usual measures of rainforests involve surveying 1km plots, measuring the diameter of all trees at a specific height and extrapolating biomass from that using allometric equations. Evidence suggests tropical forests are continuing to sequester carbon and so the question arises: does protecting carbon dense forest also protect biodiverse forest? If this was the case its a win-win however Martins study suggests that there is no clear link - even when accounting for climate/soil and spatial autocorrelation.
