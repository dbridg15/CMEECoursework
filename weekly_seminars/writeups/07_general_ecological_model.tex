\section{General Ecosystem Models: mechanistic ecology linking scales from individuals to ecosystems.}

\textbf{Speaker:} Dr Mike Harfoot (\textit{UN Environment; WCMC})\\
\textbf{Date:} \hspace{.53cm} 7\textsuperscript{th} December 2017
\vspace{.5cm}

Biodiversity is complex and fundamental to a habitable planet. Due to the complexity of moset conservation based models it is very difficult to unpick the mechanisms behind them and transfer the models to other systems. Climate and weather are similarly complex (though not quite on the same scale) and we have been developing mechanistic models around them for the last 60 years - with incredibly large amounts of data being collected. In ecology we are in the 1950's relative to climate models -- but we have to start off with gross oversimplifications.

The Madingley Model is a General Ecosystem Model which aims to explain the mechanisms which underpin how ecosystems on land and in the sea function and are structured. It attempt to consider all trophic levels and provide reproducible results. The model has realistic geography with continents, ocean circulation and environmental conditions. The aim is to run the model at the individual level -- looking at functional traits rather than specific species. Individuals with similar functional traits are grouped into cohorts and abundances calculated. What emerges is broadly similar to reality (though very simplistic). It's success varies with geography as well.

They are using the Madingley Model to predict response to Global environmental changes -- combining it with the PREDICTS database (which looks at impacts of humans on species/ecosystems). They have found that different ecosystems have quantitatively different responses. Larger organisms tend to respond faster. With this research they are able to look at the biodiversity planetary boundary and determine tipping points in environmental change beyond which the planet will be unable to recover.