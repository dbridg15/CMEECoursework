\section{Whole-genome sequencing sheds light on patterns and processes of sympatric speciation in Neotropical Midas cichlid fish.}

\textbf{Speaker:} Dr Andreas F. Kautt (\textit{University of Konstanz})\\
\textbf{Date:} \hspace{.53cm} 12\textsuperscript{th} April 2018
\vspace{.5cm}

\textit{"Sympatric speciation is like the measles, everyone gets it and we all get over it"}

\vspace{.5cm}

Speciation is the build up of inherent barriers to gene flow, traditionally a geographic barrier has seemed like an obvious instigator of speciation, however an understanding of sympatric speciation - divergence from a single ancestor in the same geographic location shows this is not necessary. Cichlid fish are an ideal candidate to study speciation, 1 in 30 vertebrates are a cichlid and their presence in crater lakes - some with flow between and others not allow for comparison of sympatric and allopatric speciation.

They looked at a number of traits which differed in fish within and between lakes and asked a number of questions to determine if divergence was occurring:
\begin{enumerate}[noitemsep]
	\item Is the divergent selection acting on the trait?
	\item Is there non-random mating?
	\item What is the genetic basis of the trait?
	\item How far in the speciation continuum is overall divergence?
	\item What does the genomic landscape look like?	
\end{enumerate}

\textbf{Lippy \textit{vs} non-lippy:} Feeding experiments on wild-caught fish showed there is a trade-off in lip size with large lips making it harder to catch evasive fish but easier to get into crevasses. Assortative mating was also found and genome studies confirmed this - it is hard however to say where along the divergence spectrum they are. There are genomic signatures around the 'lip-locus'.

\textbf{Benthic \textit{vs} Limnetic:} Where benthic fish live at the bottom of the lakes and limnetic fish stay at the top layer. There is strong evidence for assortative mating - which from experiments does not seem to be due to habitat isolation.