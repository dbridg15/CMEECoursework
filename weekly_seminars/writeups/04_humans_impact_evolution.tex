\section{Adaption in a changing world: Human influences on evolution.}

\textbf{Speaker:} Dr Kiyoko Gotanda (\textit{University of Cambridge})\\
\textbf{Date:} \hspace{.53cm} 2\textsuperscript{nd} November 2017
\vspace{.5cm}

The biodiversity which we see in the world is due to a large amount of adaptive divergence - where selection chooses for the peak fitness under different circumstances. Previous studies have assumed that selection is the same through time and populations, in reality selection is highly variable in time and between populations. Humans have a significant impact on the distribution of these selection pressures.

Does hunting affect adaptive behaviour? In Barbados fish in protected areas (where spear fishing is not allowed) were more tolerant of humans and had a shorter flight initiation distance (FID) - the distance that prey flee - as well as other decreased anti-predator behaviours.

In the Galapogos human presence may be impacting the evolution of Darwins finches. There are 14 recognized species with adaptive divergence between islands. Humans are changing the niche characteristics of these birds with populations living closer to populated areas having a greater niche overlap - as they do not need specialised beaks to eat human food, while further from town populations of different species become more specialised - reducing competition between species.

Urbanisation has also impacted the anti-predator behaviour. one hypothesis was that with a greater concentration of predators in town (where the food is) a faster anti-predator response should be expected. However the opposite was found with reduced anti-predator behaviour in town, possibly due to populations becoming more accustomed to the noise and disturbance. 

Invasive species are another impacting factor on adaption with the galapogos being ideal to study as some islands are pristine while others have feral cats and rats and others have had invasive populations eradicated. They found they found that islands with invasive predators had an increased FID with no reversion on islands which had since had the populations eradicated.